\documentclass{beamer}

\usepackage[spanish]{babel}
\usepackage{amsmath}
\usepackage{color}
%\usepackage{minted}
\usepackage{cprotect}
\usepackage{hyperref}
\usepackage{multicol}
\usepackage{tabularx}
\usepackage{tikz}

% only inline todonotes work
\usepackage{xkeyval}
\usepackage[textsize=small]{todonotes}
\presetkeys{todonotes}{inline}{}

\usetikzlibrary{shapes,arrows,positioning,shadows}

% no nav buttons
\usenavigationsymbolstemplate{}

\newcommand{\bftt}[1]{\textbf{\texttt{#1}}}
\newcommand{\comment}[1]{{\color[HTML]{008080}\textit{\textbf{\texttt{#1}}}}}
\newcommand{\cmd}[1]{{\color[HTML]{008000}\bftt{#1}}}
\newcommand{\bs}{\char`\\}
\newcommand{\cmdbs}[1]{\cmd{\bs#1}}
\newcommand{\lcb}{\char '173}
\newcommand{\rcb}{\char '175}
\newcommand{\cmdbegin}[1]{\cmdbs{begin\lcb}\bftt{#1}\cmd{\rcb}}
\newcommand{\cmdend}[1]{\cmdbs{end\lcb}\bftt{#1}\cmd{\rcb}}


% this is where the example source files are loaded from
% do not include a trailing slash
\newcommand{\fileuri}{https://raw.githubusercontent.com/joanby/curso-latex/master}

\newcommand{\wlserver}{https://es.overleaf.com}
\newcommand{\wlnewdoc}[1]{\wlserver/docs?snip\_uri=\fileuri/#1\&splash=none}

\def\tikzname{Ti\emph{k}Z}

% from http://tex.stackexchange.com/questions/5226/keyboard-font-for-latex
\newcommand*\keystroke[1]{%
  \tikz[baseline=(key.base)]
    \node[%
      draw,
      fill=white,
      drop shadow={shadow xshift=0.25ex,shadow yshift=-0.25ex,fill=black,opacity=0.75},
      rectangle,
      rounded corners=2pt,
      inner sep=1pt,
      line width=0.5pt,
      font=\scriptsize\sffamily
    ](key) {#1\strut}
  ;
}
\newcommand{\keystrokebftt}[1]{\keystroke{\bftt{#1}}}

\title{Introducción a \LaTeX{} con Overleaf(*)} 
\author{J.G Gomila y L. Valverde}
%\titlegraphic{%
%\includegraphics[height=36pt]{overleaf}\\[1em]
%%\includegraphics[height=24pt]{UoB-logo}
%}

\subtitle{Primera parte: Fundamentos}
\date{}
\begin{document}

%%%%%%%%%%%%%%%%%%%%%%%%%%%%%%%%%%%%%%%%%%%%%%%%%%%%%%%%%%%%%%%%%%%%%%%%%%%%%%%
%%%%%%%%%%%%%%%%%%%%%%%%%%%%%%%%%%%%%%%%%%%%%%%%%%%%%%%%%%%%%%%%%%%%%%%%%%%%%%%
%%%%%%%%%%%%%%%%%%%%%%%%%%%%%%%%%%%%%%%%%%%%%%%%%%%%%%%%%%%%%%%%%%%%%%%%%%%%%%%
\begin{frame}
\titlepage

{\scriptsize{(*) Basado en ``An interactive introduction to \LaTeX'' de J.D. Lees -Miller}}
\end{frame}

%%%%%%%%%%%%%%%%%%%%%%%%%%%%%%%%%%%%%%%%%%%%%%%%%%%%%%%%%%%%%%%%%%%%%%%%%%%%%%%
%%%%%%%%%%%%%%%%%%%%%%%%%%%%%%%%%%%%%%%%%%%%%%%%%%%%%%%%%%%%%%%%%%%%%%%%%%%%%%%
%%%%%%%%%%%%%%%%%%%%%%%%%%%%%%%%%%%%%%%%%%%%%%%%%%%%%%%%%%%%%%%%%%%%%%%%%%%%%%%
\begin{frame}{\LaTeX{}, ¿por qué?}
\begin{itemize}
\item Permite crear  documentos magníficos.
\begin{itemize}
\item Especialmente matemáticos
\end{itemize}
%
\item Fué creado por matemáticos, para matemáticos
\begin{itemize}
\item Una comunidad grande y activa
\end{itemize}
%
\item Es muy potente --- lo puedes extender
\begin{itemize}
\item Paquetes para artículos, presentaciones, hojas de cálculo, \ldots
\end{itemize}
\end{itemize}
\end{frame}

%%%%%%%%%%%%%%%%%%%%%%%%%%%%%%%%%%%%%%%%%%%%%%%%%%%%%%%%%%%%%%%%%%%%%%%%%%%%%%%
%%%%%%%%%%%%%%%%%%%%%%%%%%%%%%%%%%%%%%%%%%%%%%%%%%%%%%%%%%%%%%%%%%%%%%%%%%%%%%%
%%%%%%%%%%%%%%%%%%%%%%%%%%%%%%%%%%%%%%%%%%%%%%%%%%%%%%%%%%%%%%%%%%%%%%%%%%%%%%%
\begin{frame}[fragile]{¿Cómo funciona?}
\begin{itemize}
\item Escribes tu documento en \texttt{texto simple} con \cmd{comandos} que describen su estructura y significado.
\item El programa \texttt{latex} procesa el texto y los comandos para conseguir un documento magníficamente formatado.

\end{itemize}
\vskip 2ex
\begin{center}

\begin{verbatim}
 Si te dan un papel pautado {\em escribe detrás}.
\end{verbatim}

\vskip 2ex
\tikz\node[single arrow,fill=gray,font=\ttfamily\bfseries,%
  rotate=270,xshift=-1em]{latex};
\vskip 2ex
\fbox{Si te dan un papel pautado {\em escribe detrás}.}


\end{center}
\end{frame}

%%%%%%%%%%%%%%%%%%%%%%%%%%%%%%%%%%%%%%%%%%%%%%%%%%%%%%%%%%%%%%%%%%%%%%%%%%%%%%%
%%%%%%%%%%%%%%%%%%%%%%%%%%%%%%%%%%%%%%%%%%%%%%%%%%%%%%%%%%%%%%%%%%%%%%%%%%%%%%%
%%%%%%%%%%%%%%%%%%%%%%%%%%%%%%%%%%%%%%%%%%%%%%%%%%%%%%%%%%%%%%%%%%%%%%%%%%%%%%%
\begin{frame}[fragile]{Más ejemplos de comandos y su resultado \ldots}


\begin{columns}
\begin{column}{4cm}
{\scriptsize
\begin{verbatim}
\begin{itemize}
\item Picos
\item Palas
\item Azadones
\end{itemize} 
\end{verbatim} 
}
\end{column}
\begin{column}{1cm}
$\implies$
\end{column}
\begin{column}{4cm}
\begin{itemize}
\item Picos
\item Palas
\item Azadones
\end{itemize} 
\end{column}
\end{columns}

\begin{columns}
\begin{column}{4cm}
{\scriptsize
\begin{verbatim}
\begin{figure}
\includegraphics{siurellb}
\end{figure}  
\end{verbatim}}
\end{column}
\begin{column}{1cm}
$\implies$
\end{column}
\begin{column}{4cm}
\begin{figure}
\includegraphics{siurellb}
\end{figure}
\end{column}
\end{columns}


\begin{columns}
\begin{column}{4cm}
{\scriptsize
\begin{verbatim}
\begin{equation}
\alpha - \beta = 3
\end{equation}
\end{verbatim}}
\end{column}
\begin{column}{1cm}
$\implies$
\end{column}
\begin{column}{4cm}
\begin{equation}
\alpha - \beta =3
\end{equation}
\end{column}
\end{columns}

\vspace{.5cm}
\hfill \tiny{La imagen es de  \url{http://siurellscanbernadinou.com/ca/464-2/}}
\end{frame}

%%%%%%%%%%%%%%%%%%%%%%%%%%%%%%%%%%%%%%%%%%%%%%%%%%%%%%%%%%%%%%%%%%%%%%%%%%%%%%%
%%%%%%%%%%%%%%%%%%%%%%%%%%%%%%%%%%%%%%%%%%%%%%%%%%%%%%%%%%%%%%%%%%%%%%%%%%%%%%%
%%%%%%%%%%%%%%%%%%%%%%%%%%%%%%%%%%%%%%%%%%%%%%%%%%%%%%%%%%%%%%%%%%%%%%%%%%%%%%%
\begin{frame}[fragile]{Cambios en el concepto de redacción}

\begin{itemize}
\item Usa los comandos para describir `lo que es' y no `que aspecto tiene'.
\item Concentrate en el contenido..
\item Deja el resto al \LaTeX{}.
\end{itemize}
\end{frame}

%%%%%%%%%%%%%%%%%%%%%%%%%%%%%%%%%%%%%%%%%%%%%%%%%%%%%%%%%%%%%%%%%%%%%%%%%%%%%%%
%%%%%%%%%%%%%%%%%%%%%%%%%%%%%%%%%%%%%%%%%%%%%%%%%%%%%%%%%%%%%%%%%%%%%%%%%%%%%%%
%%%%%%%%%%%%%%%%%%%%%%%%%%%%%%%%%%%%%%%%%%%%%%%%%%%%%%%%%%%%%%%%%%%%%%%%%%%%%%%
\section{Fundamentos}

%%%%%%%%%%%%%%%%%%%%%%%%%%%%%%%%%%%%%%%%%%%%%%%%%%%%%%%%%%%%%%%%%%%%%%%%%%%%%%%
%%%%%%%%%%%%%%%%%%%%%%%%%%%%%%%%%%%%%%%%%%%%%%%%%%%%%%%%%%%%%%%%%%%%%%%%%%%%%%%
%%%%%%%%%%%%%%%%%%%%%%%%%%%%%%%%%%%%%%%%%%%%%%%%%%%%%%%%%%%%%%%%%%%%%%%%%%%%%%%
\subsection{Para empezar}
\begin{frame}[fragile]{\insertsubsection}
\begin{itemize}
\item Un documento mínimo de \LaTeX{}:
\cprotect\fbox{
\begin{minipage}[c]{8cm}
{\scriptsize
\begin{verbatim}
\documentclass{article}
\begin{document}
Decíamos ayer ...  % Tu contenido va aquí ...
\end{document}
\end{verbatim}}
\end{minipage}}
\item Los comandos empiezan  con una \emph{barra invertida} \keystrokebftt{\bs}.
\item Cada documento empieza con el comando \cmdbs{documentclass}.
\item El \emph{argumento} entre llaves \keystrokebftt{\{} \keystrokebftt{\}} le dice a \LaTeX{} que tipo de documento estamos creando: un \bftt{artículo}.
\item El signo \keystrokebftt{\%} empieza un \emph{comentario} --- \LaTeX{}
ignorará el resto de la línea.
\end{itemize}
\end{frame}

%%%%%%%%%%%%%%%%%%%%%%%%%%%%%%%%%%%%%%%%%%%%%%%%%%%%%%%%%%%%%%%%%%%%%%%%%%%%%%%
%%%%%%%%%%%%%%%%%%%%%%%%%%%%%%%%%%%%%%%%%%%%%%%%%%%%%%%%%%%%%%%%%%%%%%%%%%%%%%%
%%%%%%%%%%%%%%%%%%%%%%%%%%%%%%%%%%%%%%%%%%%%%%%%%%%%%%%%%%%%%%%%%%%%%%%%%%%%%%%
\begin{frame}[fragile]{Para empezar con Overleaf}
\begin{itemize}
\item Overleaf es un sitio web para escribir documentos en \LaTeX.
\item `Compila' tu documento \LaTeX{} automáticamente y te enseña el resultado.
\vskip 2em
\begin{center}
\fbox{\href{\wlnewdoc{basics.tex}}{%
Clica aquí para abrir el document de ejemplo en Overleaf}}
\\[1ex]\scriptsize{}
Obtendrás  resultados mejores si usas \href{http://www.google.com/chrome}{Google Chrome} o un \href{http://www.mozilla.org/en-US/firefox/new/}{FireFox} reciente.
\end{center}
\vskip 2ex
\item A medida que vayamos avanzando, deberías ir probando los ejemplos que vayan saliendo.
\end{itemize}
\end{frame}

%%%%%%%%%%%%%%%%%%%%%%%%%%%%%%%%%%%%%%%%%%%%%%%%%%%%%%%%%%%%%%%%%%%%%%%%%%%%%%%
%%%%%%%%%%%%%%%%%%%%%%%%%%%%%%%%%%%%%%%%%%%%%%%%%%%%%%%%%%%%%%%%%%%%%%%%%%%%%%%
%%%%%%%%%%%%%%%%%%%%%%%%%%%%%%%%%%%%%%%%%%%%%%%%%%%%%%%%%%%%%%%%%%%%%%%%%%%%%%%



\subsection{Escribir texto}
\begin{frame}[fragile]{\insertsubsection{}}
\small
\begin{itemize}
\item Escribe el texto entre \cmdbegin{document} y \cmdend{document}.
\item Para la mayor parte del documento, será sufieciente escribir el texto normalmente.
\vspace{.5cm}
\begin{center}
\cprotect\fbox{
\begin{minipage}[l]{5cm}
\begin{verbatim}
Las palabras van  separadas 
    por un espacio o más.

Los párrafos van separados 
por una o más líneas 
en blanco.

\end{verbatim}
\end{minipage}}
\fbox{\begin{minipage}[r]{4cm}
Las palabras van     separadas 
por un espacio o más.

\vspace{.4cm}
Los párrafos van separados 
por una o más líneas 
en blanco.
 \vspace{.5cm}
\end{minipage}}
\end{center}

\item Los espacios del archivo fuente se compactan en el archivo de salida: 
\begin{center}
\cprotect\fbox{
\begin{minipage}[l]{5cm}
\begin{verbatim}
Si te dan   un papel
pautado,       escribe 
detrás.
\end{verbatim}
\end{minipage}}
\fbox{\begin{minipage}[r]{4cm}
Si te dan   un papel
pautado,       escribe 
detrás.
\vspace{.35cm}
\end{minipage}}
\end{center}
\end{itemize}
\end{frame}


%%%%%%%%%%%%%%%%%%%%%%%%%%%%%%%%%%%%%%%%%%%%%%%%%%%%%%%%%%%%%%%%%%%%%%%%%%%%%%%
%%%%%%%%%%%%%%%%%%%%%%%%%%%%%%%%%%%%%%%%%%%%%%%%%%%%%%%%%%%%%%%%%%%%%%%%%%%%%%%
%%%%%%%%%%%%%%%%%%%%%%%%%%%%%%%%%%%%%%%%%%%%%%%%%%%%%%%%%%%%%%%%%%%%%%%%%%%%%%%
\begin{frame}[fragile]{\insertsubsection{}: aclaraciones}
\small
\begin{itemize}
\item Las comillas tienen su truco: Teneis que usar un acento grave \keystroke{\`{}} a la izquierda y un apóstrofe \keystroke{\'{}} a la derecha.

\begin{center}
\cprotect\fbox{
\begin{minipage}[l]{5cm}
\includegraphics[scale=0.5]{cometes}
\end{minipage}}
\fbox{\begin{minipage}[r]{4cm}
Comillas simples: `texto'.

\vspace{.4cm}
Comillas dobles: ``texto''.
\end{minipage}}
\end{center}

\item Algunos carácteres tienen un significado especial en \LaTeX:\\[1ex]
\begin{tabular}{cl}
\keystrokebftt{\%} & tanto por ciento              \\
\keystrokebftt{\#} & tablillas \\
\keystrokebftt{\&} & signo and                \\
\keystrokebftt{\$} & dólar               \\
\end{tabular}
\item Si escribes uno, obtendrás un error. Si lo necesitas, tienes que escribirlo precedido de  barra invertida:
\begin{center}

\begin{tabular}{|c|c|}
\hline
\verb+\$ \% \& \# !+ &\$ \% \&  \# ! \\
\hline
\end{tabular}
\end{center}
\end{itemize}
\end{frame}


%%%%%%%%%%%%%%%%%%%%%%%%%%%%%%%%%%%%%%%%%%%%%%%%%%%%%%%%%%%%%%%%%%%%%%%%%%%%%%%
%%%%%%%%%%%%%%%%%%%%%%%%%%%%%%%%%%%%%%%%%%%%%%%%%%%%%%%%%%%%%%%%%%%%%%%%%%%%%%%
%%%%%%%%%%%%%%%%%%%%%%%%%%%%%%%%%%%%%%%%%%%%%%%%%%%%%%%%%%%%%%%%%%%%%%%%%%%%%%%
\begin{frame}[fragile]{Gestión de los errores}
\begin{itemize}
\item \LaTeX{} puede tener problemas cuando intente compilar un documento. Si esto pasa, se para y da un mensaje de error, que has de intentar corregir antes de seguir, ya que no dará resultado alguno.

\item Por ejemple, si escribes \cmdbs{meph} en lugar de \cmdbs{emph}, el \LaTeX{} se parará con un mensaje
de error ``undefined control sequence'', dado que ``meph''  no es uno de los comandos que conoce.

\end{itemize}
\begin{block}{Consejos sobre los errores}
\begin{enumerate}
\item ¡No te asustes! Los errores ocurren.
\item Corrígelos tan tan pronto como aparezcan --- si lo que acabas de escribir es la causa del error, ya sabes por donde puedes empezar a depurar. 
\item Si hay más de un error, empieza por el primero --- la causa puede estar incluso antes que este. 
\end{enumerate}
\end{block}
\end{frame}


%%%%%%%%%%%%%%%%%%%%%%%%%%%%%%%%%%%%%%%%%%%%%%%%%%%%%%%%%%%%%%%%%%%%%%%%%%%%%%%
%%%%%%%%%%%%%%%%%%%%%%%%%%%%%%%%%%%%%%%%%%%%%%%%%%%%%%%%%%%%%%%%%%%%%%%%%%%%%%%
%%%%%%%%%%%%%%%%%%%%%%%%%%%%%%%%%%%%%%%%%%%%%%%%%%%%%%%%%%%%%%%%%%%%%%%%%%%%%%%
\begin{frame}[fragile]{Ejercicio 1 de escritura}

\begin{block}{Escribe esto en \LaTeX:
\footnote{\url{http://en.wikipedia.org/wiki/Economy_of_the_United_States}}}
In March 2006, Congress raised that ceiling an additional \$0.79 trillion to
\$8.97 trillion, which is approximately 68\% of GDP. As of October 4, 2008, the
``Emergency Economic Stabilization Act of 2008'' raised the current debt ceiling
to \$11.3 trillion.
\end{block}
\vskip 2ex
\begin{center}
\fbox{\href{\wlnewdoc{basics-exercise-1.tex}}{%
Clica para abrir este ejercicio en Overleaf}}
\end{center}

\begin{itemize}
\item Indicación: ¡vigila los carácteres con significados especiales! 
\item Una vez que lo hayas probado
\fbox{\href{\wlnewdoc{basics-exercise-1-solution.tex}}{%
clica aquí para ver una solución.}}
\end{itemize}
\end{frame}


%%%%%%%%%%%%%%%%%%%%%%%%%%%%%%%%%%%%%%%%%%%%%%%%%%%%%%%%%%%%%%%%%%%%%%%%%%%%%%%
%%%%%%%%%%%%%%%%%%%%%%%%%%%%%%%%%%%%%%%%%%%%%%%%%%%%%%%%%%%%%%%%%%%%%%%%%%%%%%%
%%%%%%%%%%%%%%%%%%%%%%%%%%%%%%%%%%%%%%%%%%%%%%%%%%%%%%%%%%%%%%%%%%%%%%%%%%%%%%%
\subsection{Expresiones matemáticas}
\begin{frame}[fragile]{\insertsubsection{}: Los signos de dólar}
\begin{itemize}
\item ¿Qué tienen de especial los signos de dólar?  \keystrokebftt{\$}?  Los usamos para marcar las fórmulas matemáticas en el texto: %\\[1ex]
\begin{center}
\cprotect\fbox{
\begin{minipage}[l]{5cm}
\begin{verbatim}
% no muy bien:
Sean a y b enteros 
positivos distintos, 
y sea c = a - b + 1.

% mucho mejor:
Sean $a$ y $b$ enteros 
positivos distintos, 
y sea  $c = a - b + 1$.
\end{verbatim}
\end{minipage}}
\fbox{\begin{minipage}[r]{4cm}

\vspace{.25cm}
Sean a y b enteros 
positivos distintos, 
y sea c = a - b + 1.

\vspace{1.3cm}
Sean $a$ y $b$ enteros 
positivos distintos, 
y sea  $c = a - b + 1$.
\end{minipage}}
\end{center}

\vspace{.5cm}
\item Los signos de dólar van siempre por parejas --- uno para abrir la expresión matemática y el otro para cerrarla..

\item \LaTeX{} gestiona los espacios automáticamente:  ignora los tuyos.
% \begin{minipage}[l]{5cm}
% \begin{verbatim}
%Sigui $y=mx+b$ \ldots
% \end{verbatim}
% \end{minipage}
%\begin{minipage}[r]{3cm}
%Sigui $y = m x + b$ \ldots
%\end{minipage}
\end{itemize}
\end{frame}


%%%%%%%%%%%%%%%%%%%%%%%%%%%%%%%%%%%%%%%%%%%%%%%%%%%%%%%%%%%%%%%%%%%%%%%%%%%%%%%
%%%%%%%%%%%%%%%%%%%%%%%%%%%%%%%%%%%%%%%%%%%%%%%%%%%%%%%%%%%%%%%%%%%%%%%%%%%%%%%
%%%%%%%%%%%%%%%%%%%%%%%%%%%%%%%%%%%%%%%%%%%%%%%%%%%%%%%%%%%%%%%%%%%%%%%%%%%%%%%
\begin{frame}[fragile]{\insertsubsection{}: Notación}
\begin{itemize}
\item Usa el acento circumflejo \keystrokebftt{\^} para los superíndices y el guión bajo para los subíndices.

\cprotect\fbox{
\begin{minipage}[l]{6cm}
\begin{verbatim}
$y = c_2 x^2 + c_1 x + c_0$ 
\end{verbatim}
\end{minipage}}
\fbox{\begin{minipage}[r]{3cm}
  $y = c_2 x^2 + c_1 x + c_0$
 \end{minipage}}
 
 \vskip 2ex

\item Usa las llaves \keystrokebftt{\{} \keystrokebftt{\}} para agrupar subíndices y superíndices.

\cprotect\fbox{
\begin{minipage}[l]{6cm}
\begin{verbatim}
$F_n = F_n-1 + F_n-2$  %epa!

$F_n = F_{n-1} + F_{n-2}$ %ok!
\end{verbatim}
\end{minipage}}
\fbox{\begin{minipage}[r]{3cm}
$F_n = F_n-1 + F_n-2$

\vspace{.5cm}
$F_n = F_{n-1} + F_{n-2}$ 
\end{minipage}}



\vskip 2ex

\item Hay comandos para las letras griegas y las notaciones habituales.

 \vspace{.5cm}
\cprotect\fbox{\begin{minipage}[l]{6cm}
\small
\begin{verbatim}
$\mu = A e^{Q/RT}$

$\Omega=\sum_{k=1}^{n}\omega_k$
\end{verbatim}
\end{minipage}}
\fbox{\begin{minipage}[r]{3cm}
$\mu = A e^{Q/RT}$

\vspace{.25cm}
$\Omega = \sum_{k=1}^{n} \omega_k$
\end{minipage}}
\end{itemize}
\end{frame}


%%%%%%%%%%%%%%%%%%%%%%%%%%%%%%%%%%%%%%%%%%%%%%%%%%%%%%%%%%%%%%%%%%%%%%%%%%%%%%%
%%%%%%%%%%%%%%%%%%%%%%%%%%%%%%%%%%%%%%%%%%%%%%%%%%%%%%%%%%%%%%%%%%%%%%%%%%%%%%%
%%%%%%%%%%%%%%%%%%%%%%%%%%%%%%%%%%%%%%%%%%%%%%%%%%%%%%%%%%%%%%%%%%%%%%%%%%%%%%%
\begin{frame}[fragile]{\insertsubsection{}: ecuaciones.}
\begin{itemize}
\item Si la fórmula es larga y asusta,  \emph{muestrala} sola en una línea, usando
\cmdbegin{equation} y \cmdend{equation}.\\[2ex]
\begin{center}
\cprotect\fbox{\begin{minipage}[l]{5cm}
{\scriptsize
\begin{verbatim}
Las raices de una ecuación 
de segundo grado vienen 
dadas por 

\begin{equation}
x = \frac{-b \pm \sqrt{b^2 - 4ac}}
         {2a}
\end{equation}
donde $a$, $b$y $c$ son \ldots
\end{verbatim}}
\vspace{.4cm}
\end{minipage}}
\fbox{\begin{minipage}[r]{4cm}
Las raices de una ecuación 
de segundo grado vienen 
dadas por
\begin{equation}
x = \frac{-b \pm \sqrt{b^2 - 4ac}}
         {2a}
\end{equation}
donde $a$, $b$ y $c$ son \ldots
\end{minipage}}
\end{center}
\vskip 1em
{\scriptsize Precaución: \LaTeX{} ignora los espacios en modo matemático, pero no puede  gestionar líneas en blanco en las ecuaciones --- no dejes líneas en blanco en ellas}
\end{itemize}
\end{frame}



%%%%%%%%%%%%%%%%%%%%%%%%%%%%%%%%%%%%%%%%%%%%%%%%%%%%%%%%%%%%%%%%%%%%%%%%%%%%%%%
%%%%%%%%%%%%%%%%%%%%%%%%%%%%%%%%%%%%%%%%%%%%%%%%%%%%%%%%%%%%%%%%%%%%%%%%%%%%%%%
%%%%%%%%%%%%%%%%%%%%%%%%%%%%%%%%%%%%%%%%%%%%%%%%%%%%%%%%%%%%%%%%%%%%%%%%%%%%%%%
\begin{frame}[fragile]{Interludio: Entornos}
\begin{itemize}
\item \bftt{equation} es un \emph{entorno} --- un contexto.
\item Un comando puede producir resultados diferentes en contextos diferentes.
\begin{center}
\cprotect\fbox{\begin{minipage}[l]{5.5cm}
{\scriptsize
\vspace{.3cm}
\begin{verbatim}
Podemos escribir
$ \Omega = \sum_{k=1}^{n} \omega_k $
en el texto, o podemos escribir
\begin{equation}
  \Omega = \sum_{k=1}^{n} \omega_k
\end{equation}
para resaltarla.
\end{verbatim}}
\vspace{.5cm}
\end{minipage}}
\fbox{\begin{minipage}[r]{3.5cm}
{\small
Podemos escribir
$ \Omega = \sum_{k=1}^{n} \omega_k $
en el texto, o podemos escribir
\begin{equation}
  \Omega = \sum_{k=1}^{n} \omega_k
\end{equation}
para resaltarla.}
\end{minipage}}
\end{center}
%\end{exampletwouptiny}
\vskip 2ex
\item Observa que el $\Sigma$ és mayor en el entorno \bftt{equation}, y como los superíndices y subíndices
cambian las posiciones, aunque hemos usado los mismos comandos.
\vskip 1em
{\scriptsize De hecho, habríamos podido escribir \bftt{\$...\$} como
\cmdbegin{math}\bftt{...}\cmdend{math}.}
\end{itemize}
\end{frame}


%%%%%%%%%%%%%%%%%%%%%%%%%%%%%%%%%%%%%%%%%%%%%%%%%%%%%%%%%%%%%%%%%%%%%%%%%%%%%%%
%%%%%%%%%%%%%%%%%%%%%%%%%%%%%%%%%%%%%%%%%%%%%%%%%%%%%%%%%%%%%%%%%%%%%%%%%%%%%%%
%%%%%%%%%%%%%%%%%%%%%%%%%%%%%%%%%%%%%%%%%%%%%%%%%%%%%%%%%%%%%%%%%%%%%%%%%%%%%%%
\begin{frame}[fragile]{Interludio: Entornos}
\begin{itemize}
\item Los comandos \cmdbs{begin} y \cmdbs{end} se usan para crear diferentes entornos. 
\vskip 2ex

\item Los entornos \bftt{itemize} y \bftt{enumerate}  generan listas.
\begin{center}
\cprotect\fbox{
\begin{minipage}[l]{5cm}
{\scriptsize
\begin{verbatim}
\begin{itemize} % no numerada 
\item Picos
\item Palas
\item Azadones
\end{itemize}

\begin{enumerate} % numerada
\item Picos
\item Palas
\item Azadones
\end{enumerate}
\end{verbatim}}
\end{minipage}}
\fbox{\begin{minipage}[r]{3cm}
\begin{itemize}
\item Picos
\item Palas
\item Azadones
\end{itemize}

\begin{enumerate} 
\item Picos
\item Palas
\item Azadones
\end{enumerate}
\end{minipage}}
\end{center}
\end{itemize}
\end{frame}


%%%%%%%%%%%%%%%%%%%%%%%%%%%%%%%%%%%%%%%%%%%%%%%%%%%%%%%%%%%%%%%%%%%%%%%%%%%%%%%
%%%%%%%%%%%%%%%%%%%%%%%%%%%%%%%%%%%%%%%%%%%%%%%%%%%%%%%%%%%%%%%%%%%%%%%%%%%%%%%
%%%%%%%%%%%%%%%%%%%%%%%%%%%%%%%%%%%%%%%%%%%%%%%%%%%%%%%%%%%%%%%%%%%%%%%%%%%%%%%
\begin{frame}[fragile]{Interludio: Paquetes}

\begin{itemize}
\item Todos los comandos y entornos que he usado hasta aquí vienen incluidos con el \LaTeX.

\item Los \emph{Paquetes} son librerías extra de comandos y entornos. Los hay a miles disponibles libremente.


\item Cada paquete que queramos usar se ha de cargar con el comando  \cmdbs{usepackage}  en el \emph{preámbulo}.

\item Ejemplo: \bftt{amsmath} de la American Mathematical Society.
\cprotect\fbox{\begin{minipage}[c]{9.5cm}
\begin{verbatim}
\documentclass{article}
\usepackage{amsmath} % preámbulo
\begin{document}
% ya puedes usar comandos amsmath aquí ...
\end{document}
\end{verbatim}
\end{minipage}}


\end{itemize}
\end{frame}


%%%%%%%%%%%%%%%%%%%%%%%%%%%%%%%%%%%%%%%%%%%%%%%%%%%%%%%%%%%%%%%%%%%%%%%%%%%%%%%
%%%%%%%%%%%%%%%%%%%%%%%%%%%%%%%%%%%%%%%%%%%%%%%%%%%%%%%%%%%%%%%%%%%%%%%%%%%%%%%
%%%%%%%%%%%%%%%%%%%%%%%%%%%%%%%%%%%%%%%%%%%%%%%%%%%%%%%%%%%%%%%%%%%%%%%%%%%%%%%
\begin{frame}[fragile]{\insertsubsection{}: Ejemplos con \bftt{amsmath}}
\begin{itemize}
{\small
\item Usa \bftt{equation*} para ecuaciones sin numerar.}

\begin{center}
\cprotect\fbox{
\begin{minipage}[l]{5cm}
{\scriptsize
\begin{verbatim}
\begin{equation*}
  \Omega = \sum_{k=1}^{n} \omega_k
\end{equation*}
\end{verbatim}}
\end{minipage}}
\fbox{\begin{minipage}[r]{3cm}
{\scriptsize
\begin{equation*}
  \Omega = \sum_{k=1}^{n} \omega_k
\end{equation*}}
\end{minipage}}
\end{center}

{\small
\item El \LaTeX{} trata las letras adyacentes como variables multiplicadas entre si, que no será siempre lo que queramos.  \bftt{amsmath} define comandos para muchos operadores matemátics habituales.}
\begin{center}
\cprotect\fbox{
\begin{minipage}[l]{5cm}
{\scriptsize
\begin{verbatim}
\begin{equation*} % ¡mal!
 min_{x,y} (1-x)^2 + 100(y-x^2)^2
\end{equation*}

\begin{equation*} % ¿bien!
\min_{x,y}{(1-x)^2 + 100(y-x^2)^2}
\end{equation*}
\end{verbatim}}
\end{minipage}}
\fbox{\begin{minipage}[r]{4cm}
{\scriptsize
\begin{equation*} % bad!
 min_{x,y} (1-x)^2 + 100(y-x^2)^2
\end{equation*}
\begin{equation*} % good!
\min_{x,y}{(1-x)^2 ++ 100(y-x^2)^2}
\end{equation*}}
\end{minipage}}
\end{center}
{\small
\item También puedes usar el comando\cmdbs{operatorname} para otros.}
\begin{center}
\cprotect\fbox{
\begin{minipage}[l]{5.5cm}
{\scriptsize
\begin{verbatim}
\begin{equation*}
\beta_i =
\frac{\operatorname{Cov}(R_i, R_m)}
     {\operatorname{Var}(R_m)}
\end{equation*}
\end{verbatim}}
\end{minipage}}
\fbox{\begin{minipage}[r]{3cm}
{\scriptsize
\begin{equation*}
\beta_i =
\frac{\operatorname{Cov}(R_i, R_m)}
     {\operatorname{Var}(R_m)}
\end{equation*}}
\end{minipage}}
\end{center}
\end{itemize}

\end{frame}


%%%%%%%%%%%%%%%%%%%%%%%%%%%%%%%%%%%%%%%%%%%%%%%%%%%%%%%%%%%%%%%%%%%%%%%%%%%%%%%
%%%%%%%%%%%%%%%%%%%%%%%%%%%%%%%%%%%%%%%%%%%%%%%%%%%%%%%%%%%%%%%%%%%%%%%%%%%%%%%
%%%%%%%%%%%%%%%%%%%%%%%%%%%%%%%%%%%%%%%%%%%%%%%%%%%%%%%%%%%%%%%%%%%%%%%%%%%%%%%
\begin{frame}[fragile]{\insertsubsection{}: Ejemplos con \bftt{amsmath}}
\begin{itemize}{\scriptsize
\item Alinea varias ecuaciones en los signos igual
\begin{align*}
(x+1)^3 &= (x+1)(x+1)(x+1) \\
        &= (x+1)(x^2 + 2x + 1) \\
        &= x^3 + 3x^2 + 3x + 1
\end{align*}
con el entorno \bftt{align*} .
\begin{center}
\includegraphics[scale=.4]{align}
\end{center}

%\fbox{\begin{minipage}[c]{8cm}
%{\scriptsize
%\begin{verbatim}
%\begin{align*}
%(x+1)^3 &= (x+1)(x+1)(x+1) \\
%             &= (x+1)(x^2 + 2x + 1) \\
%             &= x^3 + 3x^2 + 3x + 1
%\end{align*}
%\end{verbatim}}
%\end{minipage}}

\item Un signo \keystrokebftt{\&} separa la columna de la izquierda (antes del signo 
$=$) de la columna de la después (después del signo $=$).
\item La barra doble invertida \keystrokebftt{\bs}\keystrokebftt{\bs} provoca un cambio de línea.
}
\end{itemize}

\end{frame}


%%%%%%%%%%%%%%%%%%%%%%%%%%%%%%%%%%%%%%%%%%%%%%%%%%%%%%%%%%%%%%%%%%%%%%%%%%%%%%%
%%%%%%%%%%%%%%%%%%%%%%%%%%%%%%%%%%%%%%%%%%%%%%%%%%%%%%%%%%%%%%%%%%%%%%%%%%%%%%%
%%%%%%%%%%%%%%%%%%%%%%%%%%%%%%%%%%%%%%%%%%%%%%%%%%%%%%%%%%%%%%%%%%%%%%%%%%%%%%%
\begin{frame}[fragile]{Ejercicio de escritura 2}

\begin{block}{Escribe en \LaTeX lo que sigue:}
Sea $X_1, X_2, \ldots, X_n$ una sucesión de variables aleatories independientes e idénticamente distribuidas con $\operatorname{E}[X_i] = \mu$ i
$\operatorname{Var}[X_i] = \sigma^2 < \infty$, y sea
\begin{equation*}
S_n = \frac{1}{n}\sum_{i}^{n} X_i
\end{equation*}
su media. Entonces, cuando $n$ tiende a infinito, las variables aleatories
$\sqrt{n}(S_n - \mu)$ convergen a una distribución normal $N(0, \sigma^2)$.
\end{block}
\vskip 2ex
\begin{center}
\fbox{\href{\wlnewdoc{basics-exercise-2.tex}}{%
Clica aquí para abrir este ejercicio en Overleaf}}
\end{center}
\begin{itemize}
\item Indicación: el comando  para  $\infty$ es \cmdbs{infty}.
\item {\scriptsize Una vez que lo hayas intentado,}
\fbox{\href{\wlnewdoc{basics-exercise-2-solution.tex}}{%
{\scriptsize clica aquí para ver la solución del autor}}}.
\end{itemize}
\end{frame}

%%%%%%%%%%%%%%%%%%%%%%%%%%%%%%%%%%%%%%%%%%%%%%%%%%%%%%%%%%%%%%%%%%%%%
%%%%%%%%%%%%%%%%%%%%%%%%%%%%%%%%%%%%%%%%%%%%%%%%%%%%%%%%%%%%%%%%%%%%%%%%%%%
%%%%%%%%%%%%%%%%%%%%%%%%%%%%%%%%%%%%%%%%%%%%%%%%%%%%%%%%%%%%%%%%%%%%%%%%%%

\begin{frame}
  \frametitle{Clases de documentos y algunos paquetes más}
Clases típicas de documentos:
{\small
\begin{itemize}
  \item \textbf{article}: usado para escribir artículos para revistas especializadas. Se puede dividir en secciones y subsecciones. 
  \item \textbf{report}: se usa para informes largos que consten de capítulos:  projectos de fin de grado, tesis doctorales o libres no muy extensos.
  \item \textbf{book}: Como su propio nombre indica: libres y otros documentos  a doble cara de características similares a libros. Documentos que han de incluir cosas como capítulos, prólego, apendices  e incluso partes.
  \item \textbf{beamer}: para hacer diapositivas para una una presentación.
\end{itemize}}
\end{frame}

\begin{frame}
  \frametitle{Clases de documentos y algunos paquetes más}
Las clases {\tt book} y {\tt report} són muy similares. Sin embargo, tienen algunas diferencias:
\begin{itemize}
  \item Por ejemplo, la clase {\tt book} hace que los capítulos empiezen siempre en una página impar, de manera que si un capítulo anterior acaba en una de estas páginas, la  siguiente quedarà en blanco y el capítulo nuevo empezarà a continuación, en una página par.
  \item Con  la clase {\tt report} esto no pasa: un capítulo nuevo empieza siempre en la página siguiente, sea impar o no. 
\end{itemize}
\end{frame}

\begin{frame}
  \frametitle{Clases de documentos y algunos paquetes más.}

\begin{itemize}
  \item Todas las clases de la lista anterior admiten opciones adicionales.
  \item Pueden ser varias opciones, y han de ir separadas por comas.
  \item Las opciones más comunes son las siguientes:
  \begin{itemize}
    \item {\tt 10pt, 11pt, 12pt ...}: establece el tamaño de la letra en el documento. Por defecte es de {\tt 10pt}.
    \item {\tt a4paper, letterpaper ...}: definen el tamaño del paper. Por defecto, el tamaño es  {\tt a4paper}. Además, se pueden especificar otros como {\tt a5paper, b5paper}, etc.
  \end{itemize}
\end{itemize}
\end{frame}

\begin{frame}
  \frametitle{Clases de documentos y algunos paquetes más.}

  \begin{itemize}
    \item {\tt twocolumn}: para componer el documento a dos columnas.
    \item {\tt landscape} : para componer el documento en forma apaisada.
    \item {\tt twoside, oneside}: especifica si se ha de generar el documento a una o dos caras. Si no se especifica nada, los de las clases {\tt article} y {\tt report} son a una cara, y los de clase {\tt book} son a dos caras.
    \item {\tt draft}: indica que se trata de un borrador, y aporta facilitades de manipulación de versiones no definitivas.
  \end{itemize}
  \begin{exampleblock}{Ejemplo de encabezamiento}\small 
    {\tt $\backslash${documentclass[12pt,landscape,letterpaper]\{article\}}}
  \end{exampleblock}

\end{frame}



%%%%%%%%%%%%%%%%%%%%%%%%%%%%%%%%%%%%%%%%%%%%%%%%%%%%%%%%%%%%%%%%%%%%%%%%%%%%%%%
%%%%%%%%%%%%%%%%%%%%%%%%%%%%%%%%%%%%%%%%%%%%%%%%%%%%%%%%%%%%%%%%%%%%%%%%%%%%%%%
%%%%%%%%%%%%%%%%%%%%%%%%%%%%%%%%%%%%%%%%%%%%%%%%%%%%%%%%%%%%%%%%%%%%%%%%%%%%%%%
\begin{frame}{Fin de la primera parte}
\begin{itemize}
\item Enhorabuena! Ya has aprendido como \ldots
\begin{itemize}
\item Escribir un  texto en \LaTeX.
\item Usar muchos comandos.
\item Gestionar los errores cuando aparezcan.
\item Escribir algunas expresiones matemáticas.
\item Usar diferentes entornos.
\item Cargar paquetes.
\item Definir diversos tipos de documentos.
\end{itemize}
\item ¡Bien!
\item En la segunda parte veremos como usar \LaTeX{} para escribir documentos estructurados con secciones, referencias cruzadas, figuras, tablas y bibliografías. ¡Nos vemos!
\end{itemize}
\end{frame}

\end{document}
