\documentclass[12pt]{article}
\usepackage[spanish]{babel}
\usepackage{url}

\begin{document}
Diez secretos para dar una buena charla científica

Tu nombre

Introducción

El texto de este ejercicio es una versión significativamente abreviada y ligeramente modificada del excelente artículo del mismo nombre de Mark Schoeberl y Brian Toon: \url {http://www.cgd.ucar.edu/cms/agu/scientific_talk.html}

Los secretos

He compilado esta lista personal de "secretos" de escuchar a oradores efectivos e ineficaces. No pretendo que esta lista sea exhaustiva, estoy seguro de que hay cosas que me he olvidado. Pero mi lista probablemente cubra alrededor del 90% de lo que necesita saber y hacer.

1) Prepara tu material con cuidado y con lógica. Cuenta una historia.

2) Practica tu charla. No hay excusas para la falta de preparación.

3) No incluyas demasiado material. Los buenos oradores tienen uno o dos puntos centrales y se apegan a ellos.

4) Evita las ecuaciones. Se dice que por cada ecuación en una charla, la cantidad de personas que la entenderán se reducirá a la mitad. Es decir, si q es el número de ecuaciones en su charla y n el número de personas que la entienden, resulta que

n = gamma (1/2) elevado a q

donde gamma es una constante de proporcionalidad.

5) Apunta solo algunos puntos de conclusión. Las personas no pueden recordar más de un par de cosas de una charla, especialmente si escuchan muchas charlas en reuniones grandes.

6) Dirígete a la audiencia, no a la pantalla. Uno de los problemas más comunes que veo es que el conferenciante habla a la pantalla del proyector.

7) Evita hacer sonidos que distraigan. Trata de evitar "Ummm" o "Ahhh" entre oraciones.

8) Pule tus gráficos. Aquí hay una lista de sugerencias para mejores gráficos:

* Utiliza letras grandes.

* Manén los gráficos simples. No muestres gráficos que no necesitarás.

* Usa color.

9) Sé amable al responder preguntas.

10) Utiliza el humor si es posible. Siempre me sorprende ver ómo incluso de una broma realmente tonta se ríe mucho en una charla científica.

\end{document}
